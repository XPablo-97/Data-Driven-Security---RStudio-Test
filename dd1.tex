% Options for packages loaded elsewhere
\PassOptionsToPackage{unicode}{hyperref}
\PassOptionsToPackage{hyphens}{url}
%
\documentclass[
]{article}
\usepackage{amsmath,amssymb}
\usepackage{iftex}
\ifPDFTeX
  \usepackage[T1]{fontenc}
  \usepackage[utf8]{inputenc}
  \usepackage{textcomp} % provide euro and other symbols
\else % if luatex or xetex
  \usepackage{unicode-math} % this also loads fontspec
  \defaultfontfeatures{Scale=MatchLowercase}
  \defaultfontfeatures[\rmfamily]{Ligatures=TeX,Scale=1}
\fi
\usepackage{lmodern}
\ifPDFTeX\else
  % xetex/luatex font selection
\fi
% Use upquote if available, for straight quotes in verbatim environments
\IfFileExists{upquote.sty}{\usepackage{upquote}}{}
\IfFileExists{microtype.sty}{% use microtype if available
  \usepackage[]{microtype}
  \UseMicrotypeSet[protrusion]{basicmath} % disable protrusion for tt fonts
}{}
\makeatletter
\@ifundefined{KOMAClassName}{% if non-KOMA class
  \IfFileExists{parskip.sty}{%
    \usepackage{parskip}
  }{% else
    \setlength{\parindent}{0pt}
    \setlength{\parskip}{6pt plus 2pt minus 1pt}}
}{% if KOMA class
  \KOMAoptions{parskip=half}}
\makeatother
\usepackage{xcolor}
\usepackage[margin=1in]{geometry}
\usepackage{graphicx}
\makeatletter
\def\maxwidth{\ifdim\Gin@nat@width>\linewidth\linewidth\else\Gin@nat@width\fi}
\def\maxheight{\ifdim\Gin@nat@height>\textheight\textheight\else\Gin@nat@height\fi}
\makeatother
% Scale images if necessary, so that they will not overflow the page
% margins by default, and it is still possible to overwrite the defaults
% using explicit options in \includegraphics[width, height, ...]{}
\setkeys{Gin}{width=\maxwidth,height=\maxheight,keepaspectratio}
% Set default figure placement to htbp
\makeatletter
\def\fps@figure{htbp}
\makeatother
\setlength{\emergencystretch}{3em} % prevent overfull lines
\providecommand{\tightlist}{%
  \setlength{\itemsep}{0pt}\setlength{\parskip}{0pt}}
\setcounter{secnumdepth}{-\maxdimen} % remove section numbering
\ifLuaTeX
  \usepackage{selnolig}  % disable illegal ligatures
\fi
\IfFileExists{bookmark.sty}{\usepackage{bookmark}}{\usepackage{hyperref}}
\IfFileExists{xurl.sty}{\usepackage{xurl}}{} % add URL line breaks if available
\urlstyle{same}
\hypersetup{
  pdftitle={DD1},
  hidelinks,
  pdfcreator={LaTeX via pandoc}}

\title{DD1}
\author{}
\date{\vspace{-2.5em}2024-01-11}

\begin{document}
\maketitle

\hypertarget{actividad-evaluable-1}{%
\subsection{Actividad Evaluable 1}\label{actividad-evaluable-1}}

\hypertarget{data-science}{%
\subsection{Data Science}\label{data-science}}

\textbf{Pregunta 1:}

\textbf{De las siguientes preguntas, clasifica cada una como
descriptiva, exploratoria, inferencia, predictiva o causal, y razona
brevemente (una frase) el porqué}

\begin{enumerate}
\def\labelenumi{\arabic{enumi}.}
\tightlist
\item
  \textbf{Dado un registro de vehículos que circulan por una autopista,
  disponemos de su marca y modelo, país de matriculación, y tipo de
  vehículo (por número de ruedas). Con tal de ajustar precios de los
  peajes, ¿Cuántos vehículos tenemos por tipo? ¿Cuál es el tipo más
  frecuente? ¿De qué países tenemos más vehículos?}
\end{enumerate}

DESCRIPTIVAS. Ya que el resultado a analizar para subir el precio viene
dado en funció de las estadísticas recibidas del registro como conjunto
global y la pregunta no busca hacer inferencias, predicciones o
establecer relaciones causales entre las variables.

\begin{enumerate}
\def\labelenumi{\arabic{enumi}.}
\setcounter{enumi}{1}
\tightlist
\item
  \textbf{Dado un registro de visualizaciones de un servicio de
  video-on-demand, donde disponemos de los datos del usuario, de la
  película seleccionada, fecha de visualización y categoría de la
  película, queremos saber ¿Hay alguna preferencia en cuanto a género
  literario según los usuarios y su rango de edad?}
\end{enumerate}

EXPLORATORIAS. El resultado viene dado en funcion de la relación de
rango de edad y género literario.

\begin{enumerate}
\def\labelenumi{\arabic{enumi}.}
\setcounter{enumi}{2}
\tightlist
\item
  \textbf{Dado un registro de peticiones a un sitio web, vemos que las
  peticiones que provienen de una red de telefonía concreta acostumbran
  a ser incorrectas y provocarnos errores de servicio. ¿Podemos
  determinar si en el futuro, los próximos mensajes de esa red seguirán
  dando problemas? ¿Hemos notado el mismo efecto en otras redes de
  telefonía?}
\end{enumerate}

INFERENCIAS. Ya que se bass en la observación de que las peticiones de
esa red acostumbran a ser incorrectas y provocar errores de servicio y
se busca inferir si el mismo efecto se ha notado en otras redes de
telefonía.

\begin{enumerate}
\def\labelenumi{\arabic{enumi}.}
\setcounter{enumi}{3}
\tightlist
\item
  \textbf{Dado los registros de usuarios de un servicio de compras por
  internet, los usuarios pueden agruparse por preferencias de productos
  comprados. Queremos saber si ¿Es posible que, dado un usuario al azar
  y según su historial, pueda ser directamente asignado a un o diversos
  grupos?}
\end{enumerate}

PREDICITVA. Ya que se assigna a uno o diversos grupos en funcion de los
datos vistos en su historial.

\textbf{Pregunta 2:}

\textbf{Considera el siguiente escenario: Sabemos que un usuario de
nuestra red empresarial ha estado usando esta para fines no relacionados
con el trabajo, como por ejemplo tener un servicio web no autorizado
abierto a la red (otros usuarios tienen servicios web activados y
autorizados). No queremos tener que rastrear los puertos de cada PC, y
sabemos que la actividad puede haber cesado. Pero podemos acceder a los
registros de conexiones TCP de cada máquina de cada trabajador (hacia
donde abre conexión un PC concreto). Sabemos que nuestros clientes se
conectan desde lugares remotos de forma legítima, como parte de nuestro
negocio, y que un trabajador puede haber habilitado temporalmente
servicios de prueba. Nuestro objetivo es reducir lo posible la lista de
posibles culpables, con tal de explicarles que por favor no expongan
nuestros sistemas sin permiso de los operadores o la dirección.}

\textbf{Explica con detalle cómo se podría proceder al análisis y
resolución del problema mediante Data Science, indicando de donde se
obtendrían los datos, qué tratamiento deberían recibir, qué preguntas
hacerse para resolver el problema, qué datos y gráficos se obtendrían, y
cómo se comunicarían estos.}

Se debe recopilar los registros de conexiones TCP de cada máquina de
cada trabajador. A continuación, se deben limpiar y preprocesar los
datos para eliminar cualquier información redundante o no relevante.
Esto puede incluir la eliminación de registros de conexiones que no sean
relevantes para el análisis, como las conexiones de red establecidas por
clientes legítimos.

Una vez que los datos se han limpiado y preprocesado, se pueden realizar
varias preguntas para identificar al usuario que ha estado usando la red
empresarial para fines no relacionados con el trabajo. Algunas preguntas
que se pueden hacer incluyen:

• ¿Qué máquinas han establecido conexiones de red con el servicio web no
autorizado?

• ¿Cuántas conexiones de red se han establecido con el servicio web no
autorizado?

• ¿Cuándo se establecieron estas conexiones de red?

• ¿Qué usuarios estaban conectados a estas máquinas en el momento en que
se establecieron las conexiones de red?

Una vez que se han respondido estas preguntas, se pueden utilizar
gráficos para visualizar los datos y ayudar a identificar al usuario que
ha estado usando la red empresarial para fines no relacionados con el
trabajo. Por ejemplo, se pueden crear gráficos de barras para mostrar el
número de conexiones de red establecidas por cada máquina, o gráficos de
líneas para mostrar la cantidad de tráfico de red generado por cada
máquina.

Finalmente, se debe comunicar los resultados del análisis a los
operadores o la dirección de la empresa. Esto puede incluir un informe
detallado que describe los hallazgos del análisis, así como
recomendaciones para prevenir futuros incidentes de este tipo. Es
importante comunicar estos resultados de manera clara y concisa, y
asegurarse de que los operadores o la dirección de la empresa comprendan
la importancia de tomar medidas para proteger la red empresarial.

\hypertarget{introducciuxf3n-a-r}{%
\subsection{Introducción a R}\label{introducciuxf3n-a-r}}

\textbf{Pregunta 1:}

\textbf{Una vez cargado el Dataset a analizar, comprobando que se cargan
las IPs, el Timestamp, la Petición (Tipo, URL y Protocolo), Código de
respuesta, y Bytes de reply.}

\textbf{1. Cuales son las dimensiones del dataset cargado (número de
filas y columnas)}

\textbf{2. Valor medio de la columna Bytes}

\textbf{Pregunta 2:}

\textbf{De las diferentes IPs de origen accediendo al servidor, ¿cuantas
pertenecen a una IP claramente educativa (que contenga ``.edu'')?}

\textbf{Pregunta 3:}

\textbf{De todas las peticiones recibidas por el servidor cual es la
hora en la que hay mayor volumen de peticiones HTTP de tipo ``GET''?}

\textbf{Pregunta 4:}

\textbf{De las peticiones hechas por instituciones educativas (.edu),
¿Cuantos bytes en total se han transmitido, en peticiones de descarga de
ficheros de texto ``.txt''?}

\textbf{Pregunta 5:}

\textbf{Si separamos la petición en 3 partes (Tipo, URL, Protocolo),
usando str\_split y el separador '' '' (espacio), ¿cuantas peticiones
buscan directamente la URL = ``/''?}

\textbf{Pregunta 6:}

\textbf{Aprovechando que hemos separado la petición en 3 partes (Tipo,
URL, Protocolo) ¿Cuantas peticiones NO tienen como protocolo
``HTTP/0.2''?}

\begin{verbatim}
##      speed           dist       
##  Min.   : 4.0   Min.   :  2.00  
##  1st Qu.:12.0   1st Qu.: 26.00  
##  Median :15.0   Median : 36.00  
##  Mean   :15.4   Mean   : 42.98  
##  3rd Qu.:19.0   3rd Qu.: 56.00  
##  Max.   :25.0   Max.   :120.00
\end{verbatim}

\hypertarget{including-plots}{%
\subsection{Including Plots}\label{including-plots}}

You can also embed plots, for example:

\includegraphics{dd1_files/figure-latex/pressure-1.pdf}

Note that the \texttt{echo\ =\ FALSE} parameter was added to the code
chunk to prevent printing of the R code that generated the plot.

\end{document}
